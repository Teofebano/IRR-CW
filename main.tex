%
%                       This is a basic LaTeX Template
%                       for the Informatics Research Review

\documentclass[a4paper,11pt]{article}
% Add local fullpage and head macros
\usepackage{head,fullpage}     
% Add graphicx package with pdf flag (must use pdflatex)
\usepackage[pdftex]{graphicx}  
% Better support for URLs
\usepackage{url}
% Date formating
\usepackage{datetime}

\newdateformat{monthyeardate}{%
  \monthname[\THEMONTH] \THEYEAR}

\parindent=0pt          %  Switch off indent of paragraphs 
\parskip=5pt            %  Put 5pt between each paragraph  
\Urlmuskip=0mu plus 1mu %  Better line breaks for URLs


%                       This section generates a title page
%                       Edit only the following three lines
%                       providing your exam number, 
%                       the general field of study you are considering
%                       for your review, and name of IRR tutor

\newcommand{\examnumber}{1234567890}
\newcommand{\field}{Usage of Financial Network Analysis in Stock Market during Covid-19 Pandemy}
\newcommand{\supervisor}{My IRR Tutor}

\begin{document}
\begin{minipage}[b]{110mm}
        {\Huge\bf School of Informatics
        \vspace*{17mm}}
\end{minipage}
\hfill
\begin{minipage}[t]{40mm}               
        \makebox[40mm]{
        \includegraphics[width=40mm]{crest.png}}
\end{minipage}
\par\noindent
    % Centre Title, and name
\vspace*{2cm}
\begin{center}
        \Large\bf Informatics Research Review \\
        \Large\bf \field
\end{center}
\vspace*{1.5cm}
\begin{center}
        \bf \examnumber\\
        \monthyeardate\today
\end{center}
\vspace*{5mm}

%
%                       Insert your abstract HERE
%                       
\begin{abstract}
        
\end{abstract}

\vspace*{1cm}

\vspace*{3cm}
Date: \today

\vfill
{\bf Supervisor:} \supervisor
\newpage

%                                               Through page and setup 
%                                               fancy headings
\setcounter{page}{1}                            % Set page number to 1
\footruleheight{1pt}
\headruleheight{1pt}
\lfoot{\small School of Informatics}
\lhead{Informatics Research Review}
\rhead{- \thepage}
\cfoot{}
\rfoot{Date: \date{\today}}
%

\section{Introduction}

In the modern world, the global financial system is the intricate web of institutions, markets, and mechanisms that facilitates the flow of capital, currencies, and financial instruments on a worldwide scale \cite{FinancialSystem}. As the backbone of the modern global economy, the global financial system connects individuals, businesses, governments, and financial entities across the world, creating a complex network of interactions among them. This system is instrumental in allocating resources, managing risks, and supporting economic growth on an international scale. Over the years, it evolves following historical developments, technological advancements, and the dynamic nature of financial markets. This system has emerged as a key driver of economic progress and a facilitator of cross-border collaborations.

However, the global financial system possesses systemic risk due to its interconnectedness and interdependencies properties among financial institutions within. Systemic risk refers to the risk of widespread or even total failure of the system because of failure of one part of the system \cite{SystemicRisk}. The cascading effect usually is triggered by a shock or a series of interconnected events, exposing the global financial system to severe economic consequences.

Covid-19 pandemic is one of the recent examples of systemic risk effects in the global financial system. The first case of Covid was identified in China in December 2019 \cite{WHOTimeline}, subsequently spreading to East Asia, Europe, and North America shortly after resulting in pessimism in the market. Based on Statista Research Department, a significant turning point occurred on February 24, 2020, when the American stock market experienced a drastic decrease as indicated in both Dow Jones index and S\&P 500 index \cite{IndexDuringCovid}. The same study shows that the trend continued and hit the rock bottom on March 9, 2020, in which S\&P 500 index witnessed a 7\% dip and triggered a stage 1 circuit breaker, the second circuit breaker triggered in the history of the USA. This created panic in the global markets. Major indexes such as the FTSE 100, Frankfurt DAX 100, and Paris CAC 40 all decreased by more than 7\%, with the Sao Paulo B3 index plummeting by 12.16\%. This collective downturn resulted in a global financial crisis. For this reason, identifying the systemic risk in the financial systems as soon as possible is crucial to mitigate or lessen the impact of the global financial crisis.

Network science has emerged as a powerful tool for identifying and understanding systemic risk within interconnected systems, as shown by Caccioli in his research \cite{Caccioli2018}. It provides a framework to model, analyze, and visualize the relationships among units within that shape the dynamics of the system. This approach offers insights into how disruptions in one part of the network can propagate, leading to systemic implications that extend far beyond individual components. These properties allow for the identification of critical nodes, vulnerable pathways, and potential cascading effects. Researchers from around the globe have utilized network attributes to do systemic risk analysis from multiple angles. In this review, we will only focus on academic works due to its unbiased and transparent nature without disregarding companies, banks, and countries' contributions towards the utilization of network science in financial systems.

The aim of this review is to showcase current network science methodologies that can be used for managing systemic risks, enhancing resilience, and contributing to the overall stability of the system. In section 2, we will include essential theoretical aspects of network science in order to familiarize readers with fundamental knowledge of the topic. Furthermore, we will examine three attributes of the network science that mainly leveraged for that purpose

\begin{enumerate}
        \item In section 3, we will deep dive into measuring centrality attributes to identify important nodes within a network. These nodes often play pivotal roles in the transmission of risks
        \item In section 4, we will examine the level of connectivity within a network that influences the speed and extent of risk transmission
        \item In section 5, we will delve into community detection as it helps identify risk spread characteristics
\end{enumerate}
Conclusions and comments related to the topic will be presented in Section 6, while the final Section will highlight possible future research regarding the same topic.

In this research review, we will only focus on the application of network science in the Stock Market around the globe. Any other financial markets and instruments are not relevant in this review.

We base this review on peer-reviewed journals, articles, and also books to ensure the credibility of this review. We also limit the literature to be not older than 2011 as it is the initial usage of network science for systemic risk analysis in financial markets.


\section{Network Science}
This section will contain
\begin{enumerate}
    \item fundamental knowledge about network science
    \item properties of network science that is useful for systemic risk analysis and the reasonings behind it
    \item example of the network in stock market, as there are multiple ways on representing the stock market into network form
\end{enumerate}

\section{Centrality Measurement Analysis}
This section will contain answer to these questions
\begin{enumerate}
    \item How centrality measurement can be used to analyze systemic risk? and how to interpret the result?
    \item What is the context of usage, advantage, and disadvantage of each type of centrality measurement?
\end{enumerate}

The centrality properties that will be reviewed are
\begin{enumerate}
    \item degree centrality
    \item betweeness centrality
    \item eigen vector centrality
\end{enumerate}

\subsection{degree centrality}
\subsection{betweeness centrality}
\subsection{eigen vector centrality}


\section{Network Connectivity Analysis}
This section will contain answer to these questions
\begin{enumerate}
    \item How network connectivity measurement can be used to analyze systemic risk? and how to interpret the result?
    \item What is the context of usage, advantage, and disadvantage of each type of network connectivity measurement?
\end{enumerate}

The network connectivity types that will be reviewed are
\begin{enumerate}
    \item giant component 
    \item clustering coefficient 
    \item topology 
\end{enumerate}

\subsection{giant component}
\subsection{clustering coefficient}
\subsection{topology}


\section{Community Detection Analysis}
This section will contain answer to these questions
\begin{enumerate}
    \item How community detection can be used to analyze systemic risk? and how to interpret the result?
    \item What is the context of usage, advantage, and disadvantage of each type of community detection algorithm?
\end{enumerate}

The community detection algorithm that will be reviewed are
\begin{enumerate}
    \item Girvann Newman Algorithm
    \item Louvain Algorithm
\end{enumerate}

\subsection{Girvann Newman Algorithm}
\subsection{Louvain Algorithm}

\section{Summary \& Conclusion}
This section will contain the summary from multiple approaches for systemic risk analysis in stock market

\section{Future Work}
This section will contain the potential future work for systemic risk analysis in stock market

%                Now build the reference list
\bibliographystyle{unsrt}   % The reference style
%                This is plain and unsorted, so in the order
%                they appear in the document.


\small
\bibliography{main}       % bib file(s).

\end{document}

