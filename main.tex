%
%                       This is a basic LaTeX Template
%                       for the Informatics Research Review

\documentclass[a4paper,11pt]{article}
% Add local fullpage and head macros
\usepackage{head,fullpage}     
% Add graphicx package with pdf flag (must use pdflatex)
\usepackage[pdftex]{graphicx}  
% Better support for URLs
\usepackage{url}
% Date formating
\usepackage{datetime}

\newdateformat{monthyeardate}{%
  \monthname[\THEMONTH] \THEYEAR}

\parindent=0pt          %  Switch off indent of paragraphs 
\parskip=5pt            %  Put 5pt between each paragraph  
\Urlmuskip=0mu plus 1mu %  Better line breaks for URLs


%                       This section generates a title page
%                       Edit only the following three lines
%                       providing your exam number, 
%                       the general field of study you are considering
%                       for your review, and name of IRR tutor

\newcommand{\examnumber}{1234567890}
\newcommand{\field}{Usage of Financial Network Analysis in Stock Market during Covid-19 Pandemy}
\newcommand{\supervisor}{My IRR Tutor}

\begin{document}
\begin{minipage}[b]{110mm}
        {\Huge\bf School of Informatics
        \vspace*{17mm}}
\end{minipage}
\hfill
\begin{minipage}[t]{40mm}               
        \makebox[40mm]{
        \includegraphics[width=40mm]{crest.png}}
\end{minipage}
\par\noindent
    % Centre Title, and name
\vspace*{2cm}
\begin{center}
        \Large\bf Informatics Research Review \\
        \Large\bf \field
\end{center}
\vspace*{1.5cm}
\begin{center}
        \bf \examnumber\\
        \monthyeardate\today
\end{center}
\vspace*{5mm}

%
%                       Insert your abstract HERE
%                       
\begin{abstract}
        When Covid-19 hits the world, it changed the landscape of stock market around the world.
        The pandemy exposed many systemic risks in global stock market, lethal enough to ripple and plummet stock prices across the globe.
        In this review, we examine the usage of Financial Network Analysis to model global crisis effect on global stock market and its application to prepare ourselves from similar disasters in the future.
\end{abstract}

\vspace*{1cm}

\vspace*{3cm}
Date: \today

\vfill
{\bf Supervisor:} \supervisor
\newpage

%                                               Through page and setup 
%                                               fancy headings
\setcounter{page}{1}                            % Set page number to 1
\footruleheight{1pt}
\headruleheight{1pt}
\lfoot{\small School of Informatics}
\lhead{Informatics Research Review}
\rhead{- \thepage}
\cfoot{}
\rfoot{Date: \date{\today}}
%

\section{Introduction}

In December 2019, the first case of Covid-19 was founded in Wuhan, China. 
Classified firstly as atypical pneumonia cases, it drew World Health Organization (WHO) attention in January 2020.
By the end of January, the Director General of WHO declared Covid-19 outbreak a Public Health Emergency of International Concern (PHEIC).  \cite{WHOTimeline} 
Unaware of the dire situation, the International Monetary Fund (IMF) had estimated 3.3\% 2020 global economy growth at the same time of emergency declaration. \cite{IMFEstimation} 
This was followed by most of the central banks in the world finishing their first policy meeting to prepare themselves for financial situations in 2020. \cite{CentralBankTimeline}
In the span of only two weeks, those central banks were forced to adapt their policies with Covid situation. The Central bank of the United States (the Feds) was even taking dramatic countermeasures by slashing down their basis rate to almost 0\% interest rate on March 15th 2020. This new policy was based on March 9th catastrophic event in the stock market, in which S\&P 500 (stock market index of 500 of the largest companies listed in United States Stock exchanges) dropped by 7\% in only 4 minutes triggering circuit breaker for the first time since the 2008 financial crisis. \cite{SAMITAS2022102005} 
The stock market free-fall events could also be observed in other stock indexes around the world in the following days, such as FTSE 100 (Europe) and Nikkei 225 (Japan) in which they were dropping 8.5\% and 5.1\% respectively on March 12th.

The domino effect observed in the stock market across the globe during Covid-19 pandemic is concrete proof of Financial Contagion, in which a financial crisis in one stock market strongly and rapidly spreads to other stock markets too. \cite{KOLLMANN2013139} 
The effect of this phenomenon was also amplified dramatically by uncertain situations during Covid-19 pandemic. By using Network Analysis, we can identify the financial connectedness of global stock exchanges and expose macroeconomic and systemic risks. \cite{RIZWAN2020101682}
The Financial Network Analysis can then be used to model global pandemic effects on stock markets, acting as financial crisis forecaster and guidelines to navigate through uncertain and risky situations for both stock exchanges and investors. 

The aim of this review is to showcase how to utilize Network Analysis in global stock markets to model Financial Contagion and reap forecasting and guidelines benefits from it. 
We will then analyze two different point of views of Network Analysis
\begin{enumerate}
        \item In section 3, we will deep dive into forecasting techniques that are based on Financial Contagion network model and emphasizing on identifying hidden and obscure patterns that are correlated to stock market
        \item In section 4, we will shift our focus on drawing guidelines from the Financial Contagion network model for stock market players in order to maneuver through the financial crisis. In this section. There will be two point of views for the guidelines, one for stock exchanges and the other one for investors
\end{enumerate}
Conclusions and comments related to the topic will be presented in Section 5, while the final Section will highlight possible future research regarding the same topic.  

In section 2, we will include theoretical aspects of Network Analysis, Network Modelling, and Financial Contagion in the Stock Market in order to familiarize reader with fundamental knowledge of the topic.
This is essential for the readers to understand the problems and grasp the magnitude of the proposed solutions.
However, we will only focus on the application of Network Analysis and Modelling in Stock Market. 
Any other financial markets and instruments are not relevant in this review.

We base this review on peer-reviewed journal, articles, and also books to ensure the credibility of this review. 
We also limit the literature to be not older than 2019 to ensure its relevance to Covid-19 topic.

\section{Literature Review}

\begin{enumerate}
    \item There are many ways to organize the evaluation of the sources. Chronological and thematic approaches are each useful examples.
    \item Each work should be critically summarized and evaluated for its premise, methodology, and conclusion. It is as important to address inconsistencies, omissions, and errors, as it is to identify accuracy, depth, and relevance.
    \item Use logical connections and transitions to connect sources.
\end{enumerate}

\section{Summary \& Conclusion}

\begin{enumerate}
    \item The conclusion summarizes the key findings of the review in general terms. Notable commonalities between works, whether favourable or not, may be included here.
    \item This section is the reviewer’s opportunity to justify a research proposal. Therefore, the idea should be clearly re-stated and supported according to the findings of the review.
\end{enumerate}


%                Now build the reference list
\bibliographystyle{unsrt}   % The reference style
%                This is plain and unsorted, so in the order
%                they appear in the document.


\small
\bibliography{main}       % bib file(s).

\end{document}

